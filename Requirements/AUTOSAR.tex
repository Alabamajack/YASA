\documentclass[a4paper,11pt]{scrreprt}
% \usepackage{ngerman}
\usepackage[ngerman]{babel}
\usepackage[utf8x]{inputenc}
\usepackage[T1]{fontenc}
\usepackage{multicol}
\usepackage{ifpdf}
\usepackage[pdftex]{color}
\usepackage{xcolor}
\usepackage{scrhack}
\usepackage{listings}
\usepackage[colorlinks=true,linkcolor=black]{hyperref}
\usepackage{geometry}
\geometry{a4paper,left=30mm,right=20mm, top=26mm, bottom=35mm}
\usepackage{enumerate}

\lstdefinestyle{customc}{
  belowcaptionskip=1\baselineskip,
  breaklines=true,
  frame=L,
  xleftmargin=\parindent,
  language=C,
  showstringspaces=false,
  basicstyle=\footnotesize\ttfamily,
  keywordstyle=\bfseries\color{green!40!black},
  commentstyle=\itshape\color{purple!40!black},
  identifierstyle=\color{blue},
  stringstyle=\color{orange},
}

\lstdefinestyle{customasm}{
  belowcaptionskip=1\baselineskip,
  frame=L,
  xleftmargin=\parindent,
  language=[x86masm]Assembler,
  basicstyle=\footnotesize\ttfamily,
  commentstyle=\itshape\color{purple!40!black},
}
\lstset{escapechar=@,style=customc}

\ifpdf
  \usepackage[pdftex]{graphicx}
\else
  \usepackage[dvips]{graphicx}\fi

\newcommand{\kur}{\textit}
\newcommand{\ul}{\underline}
\newcommand{\bo}{\textbf}

\setcounter{tocdepth}{3}

\newcommand{\format}{\textbf}

\begin{document}

\begin{titlepage}

\vspace*{\fill}
  \begin{center}

\huge \bfseries AUTOSAR \\[2.5cm]

\textsc{\Large SoSe 2015}\\[0.5cm]

\large \today

\vfill

  \end{center}
\end{titlepage}

\begin{itemize}
\item[] \textbf{\large Beitragende:}\\
Daniel Tatzel (DT)\\
Florian Laufenböck (FL)\\
Markus Wildgruber (MW)\\
Philipp Eidenschink (PE)\\
Tim Schmiedl (TimS)\\
Tobias Schwindl (TobiS)
\end{itemize}

\bigskip

\begin{table}[!h]
 	\centering
	\begin{tabular}{|c|c|c|c|}
	\hline
	\textbf{VersionsNr} &  \textbf{Datum} & \textbf{Auslöser} & \textbf{Beschreibung} \\
	\hline
	1.0 & 21.04.2015 & DT & Erster Entwurf \\
	1.1 & 7.06.2015 & & Überarbeitung/Funktionsapi\\
	\hline
	\end{tabular}

% \caption{Überarbeitungshistorie}
\end{table}



\chapter{Projekt Beschreibung}

\section{Vernetzte Ballschussanlage}

\begin{itemize}
 \item 1-2 Bricks
 \item Ausgabe(durch Display,LEDs etc.)
 \item Stop-Trigger
 \item Variable Aufteilung unter den Bricks: Stopp-Taste, Auslösung Taste(auch über Ultraschall), Ausgabe
\end{itemize}

\section{Benötigte VFB-Komponenten und Schnittstellen (DT)}

\begin{itemize}
 \item Komponenten
 \begin{itemize}
  \item Application Software Component
  \item Sensor-Actuator Software Component
  \item ECU Abstraction Software Component
 \end{itemize}

 \item Schnittstellen
 \begin{itemize}
  \item Client/Server
  \item Events
  \item Sender/Receiver (auch mit synchronisierung)
 \end{itemize}

\end{itemize}


\section{Namenskonventionen und Standardrückgabtyp (Alle)}

\begin{tabular}{ll}
 Für RTE-Funktionen: & RTE\_<Komponentenname>\_<Funktionsname>\_<Portname>\_<Direction> \\
 Für den Rest: & <Komponente>\_<Funktionsname> \\
 \\
 Standardrückgabtyp: & uint32\_t $\equiv$ std\_return \\
\end{tabular}

\begin{figure}[htbp]
 \centering
 \includegraphics[scale=.75]{./KomponentenDiagramm.pdf}
 \label{fig:komponentendiag}
 \caption{Komponentendiagramm der Ballschussanlage (DT)}
\end{figure}

\section{Komponenten Funktionsapi}
Die Funktionsprototypen für die RTE-Funktionen sind in der Codedokumentation zu finden(unter \textit{YASA\_RTEAPI.h}).

\chapter{Komponenten-Beschreibung}

\section{Lose Beschreibung}

\subsection*{Schussanlage (FL)}

\begin{itemize}
	\item Besteht aus einer Task mit zwei Runnables
	\item erste Runnable prüft periodische die Abbruchbedinung(hier: Taster)
	\item zweite Runnable managt den Schussmotor
	\item Kein Autostart des Tasks, wird über den Trigger gestartet
	\item Ports siehe Komponentendiagramm
\end{itemize}

Benötigt: Task und Event

\subsection*{Trigger (PE)}

\begin{itemize}
 \item Ein Task
 \item Wird zu beginn gestartet (Autostart)
 \item Wartet auf Event vom Input
\end{itemize}

Benötigt: Task und Event

\subsection*{Output (MW)}

\begin{itemize}
 \item Autostart
 \item Wird durch Event von Schussanlage getriggert
 \item Prüft nach Event die empfangene Nachricht
 \item Zeigt Nachricht in Abhängigkeit der empfangen Nachricht an
\end{itemize}

Benötigt: Task und Event


\subsection*{SchussMotor (TimS)}

\begin{itemize}
 \item Kein Autostart
 \item Servertask wird durch Schussanlage (client) gestartet
 \item Steuert Motor zum schießen an
\end{itemize}


\subsection*{StopSensor (TobiS)}

\begin{itemize}
 \item Autostart
 \item Prüft Taster
 \item Setzt Event für Schussanlage
\end{itemize}

Benötigt: Task, Timer und Event


\subsection*{StartTrigger (TobiS)}

\begin{itemize}
 \item Task zum Erkennen von Zielen
 \item Autostart
 \item Sendet Event an Trigger
 \item Erkennung durch periodische Abfrage
\end{itemize}

Benötigt: Task und Timer

\section{Architekturschicht und Funktionsapi}

\includegraphics{Komponenten.png}

\subsection{Funktionapi}
\begin{enumerate}[1.)]
\item System Services\\
	keine Funktionen
\item Communication Services
	\begin{itemize}
	\item \underline{Abstraktionsebene um Nachrichten zu verschicken}
	\item \lstinline|StdReturnType TransmitMessage(char* message)|
	\item \lstinline|StdReturnType ReceiveMessage(char* message)|
	\end{itemize}
\item I/O Hardware Abstraction
	\begin{itemize}
	\item \lstinline|StdReturnType ReadDigitalInput(PortName)|
	\item \lstinline|StdReturnType ReadAnalogInput(PortName)|
	\item \lstinline|StdReturnType DriveMotor(Port_Name, Direction, speed, angle)|
	\end{itemize}
\item Communication Hardware Abstraction
	\begin{itemize}
	\item \underline{Für unser Projekt eigentlich unnötig, da wir nur eine Kommunikationsebene haben}(theoretisch mehr durch I2C, aber hier uninteressant)
	\item \lstinline|StdReturnType SendMessageBT(char* message )|
	\item \lstinline|StdReturnType GetMessageBT(char* message)|
	\end{itemize}
\item Communication Drivers
	\begin{itemize}
	\item \underline{es wird nur ein Treiber für das Hardware BT gebraucht:}
	\item \lstinline|StdReturnType BT_Write(char* message)| 
	\item \lstinline|StdReturnType BT_Read(char* message)|
	\end{itemize}
\item I/O Drivers
	\begin{itemize}
	\item \underline{benötigt für den zusätzlichen I2C expander}
	\item \lstinline|StdReturnType ReadI2C(PortName)|
	\item \lstinline|StdReturnType WriteI2C(PortName)|	
	\end{itemize}
\end{enumerate}

\section{Versendete Nachrichten}

\chapter{Konventionen/Definitionen}
\section{Tasks}
\subsection{impliziter Task}
Pro Brick gibt es einen impliziten Task, der immer bei der Codegenerierung mithineingeneriert wird. Dieser Task ist für die Nachrichtenabholung, die per Bluetooth eintreffen, zuständig. Dieser Task hat nur eine Runnable. In der Runnable ist nichts zu tun, ausser auf Nachrichten zu warten und wenn eine Nachricht eintritt den Communicationservice aufzurufen/zu informieren.

\end{document}

